%  LaTeX support: latex@mdpi.com 
%  For support, please attach all files needed for compiling as well as the log file, and specify your operating system, LaTeX version, and LaTeX editor.

%=================================================================
\documentclass[entropy,article,submit,pdftex,moreauthors]{Definitions/mdpi} 

%=================================================================
% MDPI internal commands - do not modify
\firstpage{1} 
\makeatletter 
\setcounter{page}{\@firstpage} 
\makeatother
\pubvolume{1}
\issuenum{1}
\articlenumber{0}
\pubyear{2024}
\copyrightyear{2024}
%\externaleditor{Academic Editor: Firstname Lastname}
\datereceived{ } 
\daterevised{ } % Comment out if no revised date
\dateaccepted{ } 
\datepublished{ } 
%\datecorrected{} % For corrected papers: "Corrected: XXX" date in the original paper.
%\dateretracted{} % For corrected papers: "Retracted: XXX" date in the original paper.
\hreflink{https://doi.org/} % If needed use \linebreak
%\doinum{}
%\pdfoutput=1 % Uncommented for upload to arXiv.org
%\CorrStatement{yes}  % For updates


%=================================================================
% Add packages and commands here. The following packages are loaded in our class file: fontenc, inputenc, calc, indentfirst, fancyhdr, graphicx, epstopdf, lastpage, ifthen, float, amsmath, amssymb, lineno, setspace, enumitem, mathpazo, booktabs, titlesec, etoolbox, tabto, xcolor, colortbl, soul, multirow, microtype, tikz, totcount, changepage, attrib, upgreek, array, tabularx, pbox, ragged2e, tocloft, marginnote, marginfix, enotez, amsthm, natbib, hyperref, cleveref, scrextend, url, geometry, newfloat, caption, draftwatermark, seqsplit
% cleveref: load \crefname definitions after \begin{document}

%=================================================================
% Please use the following mathematics environments: Theorem, Lemma, Corollary, Proposition, Characterization, Property, Problem, Example, ExamplesandDefinitions, Hypothesis, Remark, Definition, Notation, Assumption
%% For proofs, please use the proof environment (the amsthm package is loaded by the MDPI class).

%=================================================================
% Full title of the paper (Capitalized)
\Title{How Information Evolves}

% MDPI internal command: Title for citation in the left column
\TitleCitation{How Information Evolves}

% Authors, for the paper (add full first names)
\Author{Dan Adler $^{1}$}

%\longauthorlist{no}

% MDPI internal command: Authors, for metadata in PDF
\AuthorNames{Dan Adler}

% MDPI internal command: Authors, for citation in the left column
\AuthorCitation{Dan Adler}
% If this is a Chicago style journal: Lastname, Firstname, Firstname Lastname, and Firstname Lastname.

% Affiliations / Addresses (Add [1] after \address if there is only one affiliation.)
\address{%
$^{1}$ \quad dan@danadler.com}

%\simplesumm{} % Simple summary

%\conference{} % An extended version of a conference paper

% Abstract (Do not insert blank lines, i.e. \\) 
\abstract{Evolution is a fundamental process of life, requiring a population of replicating entities capable of adaptation, governed by fitness-driven selection. In this paper, we explore a form of abiotic evolution, demonstrating how probability imbalance, feedback loops, and population dynamics can drive the emergence of ordered patterns without reliance on specific physical laws. Through abstract, non-physical systems of symbolic tokens governed by local probabilistic interactions, we show how imbalances in frequency and stability create feedback loops that produce emergent complexity. We discuss how these mechanisms mix top-down and bottom-up causality. Our findings provide insights into the principles of pattern formation across biological, chemical, and abstract systems. This study highlights that emergent complexity is a universal property of systems with probabilistic interactions and selection mechanisms, offering a new perspective on the origins of pattern formation beyond the constraints of physical determinism.}

% Keywords
\keyword{Abiotic evolution; emergence; information theory; entropy; randomness; pattern formation; universal evolution;  stability; feedback loops; Bayesian reasoning; top-down causality; symbolic systems; complex systems}

\graphicspath{{images/}}

%%%%%%%%%%%%%%%%%%%%%%%%%%%%%%%%%%%%%%%%%%
\begin{document}
%%%%%%%%%%%%%%%%%%%%%%%%%%%%%%%%%%%%%%%%%%

% The order of the section titles is different for some journals. Please refer to the "Instructions for Authors” on the journal homepage.

\section{Introduction}

This paper considers "ABC Systems" defined as follows: a population of elements \( A, B, C, \dots \) capable of forming compounds through local interactions of unspecified forces. These elements could exist in our physical universe, or in an abstract universe. The stability of compounds determines how many generations they will persist, and we assume the base elements regenerate. Over successive generations, the system undergoes a form of natural selection where compound stability drives the evolution of the population. ABC systems are governed by two distinct but interrelated probability distributions: The population distribution, reflecting the relative abundance of patterns in the system, and the stability distribution, representing the likelihood that interactions between patterns are successful and result in new, stable patterns. Together, these distributions determine the dynamics of the system, including the selection of stable patterns, their evolution over generations, and the emergent information encoded within the system.

Consider a population of 3 elements: {A, B, C}. Assume B-compounds are more stable than those without B. Thus AB, BC, ABC, have a higher stability than AC or A or C. Therefore, B-compounds will persist over multiple generations, while the others will quickly dissipate. The more stable B-compounds will interact more frequently due to their relative frequency, even without replication or inheritance. A simulation may look like this:

\begin{figure}[htp]
    \centering
    \includegraphics[width=13cm]{pat_1}
    \caption{ABC System population evolution simulation}
    \label{fig:pat_1}
\end{figure}

Where is Maxwell's demon hiding in this example, driving it towards a low-entropy state? The answer lies in the roulette wheel. Compounds that persist longer have more chances to interact. As their frequency in the population grows, their chances to interact grow even more. In Evolutionary Dynamics, this is called fitness-proportionate selection or roulette wheel selection.

\section{Dual Probability Distributions Governing ABC Systems}

ABC systems can be effectively described by two interdependent probability distributions: the \textbf{population distribution}, which represents the relative abundance of patterns in the system, and the \textbf{stability distribution}, which quantifies the likelihood of successful interactions between patterns and the resulting stability of those interactions. These distributions interact to drive the system's dynamics, determining which patterns persist, dominate, or dissipate over time.

The population distribution reflects the probabilities of patterns at any given generation \( t \). Let \( P(p_t) \) represent the probability of pattern \( p \) in the population at time \( t \):
\[
P(p_t) = \frac{N(p_t)}{\sum_{q \in P} N(q_t)},
\]
where \( N(p_t) \) is the absolute count of pattern \( p \) in the population at time \( t \), and \( \sum_{q \in P} N(q_t) \) is the total population size at time \( t \), ensuring normalization (\( \sum_{p \in P} P(p_t) = 1 \)). The population distribution evolves over generations as patterns interact and new stable configurations emerge. As patterns with higher stability and frequent interactions become more prominent in the population, the population distribution evolves to reflects the history of past interactions and selection pressures.

The stability distribution represents the likelihood of successful interactions between patterns \( p \) and \( q \). Let \( S(p, q) \) denote the stability of the interaction between \( p \) and \( q \), defined as:
\[
S(p, q) = P(r | p, q),
\]
where \( P(r | p, q) \) is the conditional probability of forming pattern \( r \) as a result of the interaction between \( p \) and \( q \).

Stability reflects inherent properties of the patterns such as physical proximity, strength of bonds, geometric compatibility, energy barriers, or interaction rules. Stability values vary for different pairs of patterns, influencing which interactions dominate the dynamics. While the population distribution evolves rapidly, the stability distribution remains constant or changes only gradually.

\subsection{Combined Role of Population and Stability Distributions}

The two distributions together govern the dynamics of the system. The probability of two patterns \( p \) and \( q \) interacting to form a new pattern \( r \) is proportional to their population probabilities and the stability of their interaction:
\[
P(p_{t+1}) \propto P(p_t) \cdot \sum_{q \in P} P(q_t) \cdot S(p, q).
\]
The continuous-time dynamics of the feedback system can be derived as a natural extension of the discrete update equation:
\[
\frac{dP(p, t)}{dt} \approx \alpha P(p, t) \cdot S_{\text{eff}},
\]
where \( S_{\text{eff}} = \sum_{q \in P} P(q, t) \cdot S(p, q) \) is the effective stability of the pattern. Order is generated in the system through feedback-driven amplification of stable patterns. The probability of a pattern \( P(p, t) \) grows exponentially over time as a result of stabilizing interactions, following:
\[
P(p, t) \propto \exp(\alpha S_{\text{eff}} t),
\]
where \( \alpha \) is the growth rate constant, and \( S_{\text{eff}} \) is the effective stability of the pattern. This exponential growth rapidly increases the dominance of stable patterns in the population.

In contrast, random processes and stochastic fluctuations redistribute probabilities in the system, contributing to an increase in entropy. The degradation of order due to these random effects is characterized by a slower logarithmic growth in entropy:
\[
\Delta H \propto \ln(t).
\]
As time progresses, the competition between the exponential growth of stable patterns and the logarithmic increase in entropy becomes evident. The net result is that the exponential growth of order, driven by stabilizing feedback and represented as \( \exp(\alpha S_{\text{eff}} t) \), outpaces the entropy increase \( \ln(t) \). This ensures that stability-driven feedback creates a net reduction in entropy, leading the system toward a more ordered, low-entropy state. Over time, the dominance of stable patterns suppresses randomness, driving the system into configurations characterized by emergent complexity and order.

This analysis demonstrates how stability-driven interactions in local populations generate order faster than the second law of thermodynamics can degrade it. While random fluctuations tend to increase entropy, the exponential reinforcement of stable patterns creates a net reduction in entropy, leading to emergent order. The dynamics also highlight the competitive nature of the system: only patterns with sufficiently high stability and effective interaction networks survive over the long term, ensuring the self-organization of the population into ordered configurations.

\section{Catalysis}

Consider an ABC system where \( A \) interacts with \( B \) to form a compound \( AB \), but this interaction occurs only in the presence of a catalyst \( C \). The catalyst \( C \) is not consumed or altered during the interaction, yet it influences the system's dynamics by facilitating specific reactions. This creates selection pressure for \( C \) to persist due to its role in enabling the formation of stable compounds. In this case, the catalyst \( C \) facilitates the reaction by increasing its likelihood:
\[
P(A + B \to AB \mid C) \propto p_A \cdot p_B \cdot p_C.
\]

The catalyst \( C \) benefits indirectly, as its presence enables reactions that increase the overall fitness of the system.

\section{Interpreting Interactions as Logic Gates in ABC Systems}

ABC systems can be interpreted as encoding logical operations through their interactions, with patterns and their combinations acting as inputs and outputs of logical gates. Catalysis, a common feature in chemical systems, further reinforces this interpretation, as catalysts act as conditional enablers for specific reactions. This section explores how interactions in ABC systems can be mapped to logical operations and how catalytic behaviors correspond to logical gates or conditional statements.

\subsection{Interactions as Logical Operations}

In an ABC system, the combination of patterns can represent logical relationships. For instance, the interaction \( A + B \to AB \) can be interpreted as a logical AND operation, where \( AB \) is produced only if both \( A \) and \( B \) are present. The probabilities of patterns encode their logical "truth values," with higher probabilities corresponding to "true" states and lower probabilities corresponding to "false" states. We already showed that interaction \( A + B \to AB \) can be expressed as:
\[
P(AB) = P(A) \cdot P(B) \cdot S(AB),
\]
where \( S(AB) \) is the Stability. This is analogous to the AND operation in logic, where the output is true only if both inputs are true. Similarly, the OR operation can be represented as the dominance of either \( A \) or \( B \), given by:
\[
P(A \lor B) = P(A) + P(B) - P(A) \cdot P(B).
\]
The logical NOT operation corresponds to the absence of a pattern, represented as:
\[
P(\neg A) = 1 - P(A).
\]
These interactions collectively encode basic logical operations, enabling the system to process and store information in a manner analogous to digital logic.

\subsection{Catalysis as Conditional Logic}

In the context of ABC systems, a catalyst \( C \) facilitating the reaction \( A + B \to AB \) can be interpreted as a conditional statement or a logical gate. The presence of \( C \) determines whether the reaction occurs, making it analogous to the logical "if" condition: "If \( C \), then \( A + B \to AB \)."
This behavior maps directly to the conditional logic often found in programming and computation. In terms of probability, the catalyzed interaction can be expressed as:
\[
P(AB | C) \propto P(A) \cdot P(B) \cdot S(A, B | C),
\]
where \( S(A, B | C) \) represents the stability of the interaction in the presence of the catalyst \( C \).

Catalysts can also be compared to the gate of a transistor in an electronic circuit. Just as a transistor gate controls the flow of current between its source and drain, a catalyst \( C \) controls the occurrence of the reaction \( A + B \to AB \). The analogy extends to the amplification effect: a small amount of \( C \) can enable significant interaction between \( A \) and \( B \), similar to how a small gate voltage modulates a larger current.

Additionally, catalysts can encode more complex logical gates. For example, a two-step catalysis process \( C_1 \) and \( C_2 \) enabling \( A + B \to AB \) and \( AB + C \to ABC \) respectively corresponds to a sequential logic operation:
\[
\text{IF } C_1, \text{ THEN } A + B \to AB, \quad \text{AND IF } C_2, \text{ THEN } AB + C \to ABC.
\]
Such catalytic networks can encode conditional logic akin to nested IF statements in programming or complex gates in digital logic circuits. 

Interactions in ABC systems can be interpreted as logical operations, with catalysts playing the role of conditional gates or enabling mechanisms. This mapping reveals a computational aspect of such systems, where reactions and stability constraints naturally encode logic. Catalytic behaviors extend this logic by introducing conditionality, enabling ABC systems to perform more complex computations. These insights bridge the gap between chemical interactions and digital computation, highlighting the potential for emergent logic and information processing in abiotic systems.


\section{Mixing Two Populations: Opportunities for Crossover and Symbiosis}

Mixing two previously separate populations introduces new interaction dynamics, enabling the formation of new stable combinations of unique compounds arising from cross-population interactions, as well as new catalytic elements enhancing reaction rates, creating pathways for more stable patterns than existed in either population individually.

Imagine two such populations: \( E_1 = \{A_1, B_1, C_1, \dots\} \), \( E_2 = \{A_2, B_2, C_2, \dots\} \) each at a different stage in its evolution. Within each population there will be compounds such as \( C_{A_1B_1} \) and \( C_{A_2B_2} \), but now there are opportunities to form new compounds across both populations, e.g., \( C_{A_1B_2}, C_{A_2B_1} \), which were previously impossible. These compounds could be more stable than any of the compounds in either population, which would create new imbalances that further reduce the entropy of the combined population. We can call this crossover, as it closely resembles genetic crossover in biotic evolution.

Furthermore, catalysts in \( E_1 \) may stabilize reactions in \( E_2 \) and vice versa, introducing symbiotic dynamics:
\[
P(A_1 + B_2 \to C_{A_1B_2} \mid D_1) \propto f_{A_1} \cdot f_{B_2} \cdot f_{D_1}.
\]

Cross-population interactions increase the diversity of compounds. For example:
\[
F(C_{A_1B_2}) = S(C_{A_1B_2}) \quad \text{and} \quad F(C_{A_2B_1}) = S(C_{A_2B_1}),
\]
where \( S(C) \) measures the stability of new compounds. These compounds contribute to the fitness landscape of both populations. Catalysts from one population (\( D_1 \)) may enable reactions in the other population (\( E_2 \)):
\[
P(A_2 + B_2 \to C_{A_2B_2} \mid D_1) \propto f_{A_2} \cdot f_{B_2} \cdot f_{D_1}.
\]
Such symbiotic relationships increase the persistence of \( D_1 \) and \( D_2 \), enhancing the overall stability of the mixed population.

Cross-population compounds and catalytic symbiosis introduce new feedback loops: the formation of \( C_{A_1B_2} \) increases \( f_{A_1} \) and \( f_{B_2} \), and the enhanced stability of \( C_{A_2B_1} \) reinforces the presence of \( A_2 \) and \( B_1 \).

\section{Analogy to Evolutionary Mechanisms}

\subsection*{Crossover in Genetics}
The formation of cross-population compounds parallels genetic crossover in biological systems, where recombination generates new traits.

\subsection*{Symbiosis in Ecosystems}
Catalytic interactions between populations resemble symbiotic relationships in ecosystems, where mutual benefits drive stability and persistence.

\section{Mathematical Properties}

\subsection*{Stationary Distribution}
If \( F(C) \) and \( g(C) \) are constant, the mixed system converges to a stationary distribution:
\[
f_X^* \quad \text{biased toward elements forming stable cross-population compounds.}
\]

\subsection*{Entropy Reduction}
The mixing of populations initially increases diversity, raising entropy. However, as stable compounds dominate, entropy decreases:
\[
H^{(t+1)} < H^{(t)}.
\]

\subsection*{Enhanced Stability}
Mixed populations can achieve greater stability than isolated populations:
\[
\max(S(C_{XY})) \text{ in mixed populations} > \max(S(C_{XY})) \text{ in isolated populations.}
\]

\section{Introduction}

Emergent phenomena in complex systems often defy simple bottom-up explanations, as the interplay between local interactions and global feedback can generate hierarchical structures and persistent patterns~\cite{anderson1972more}. This paper explores a framework for the emergence of stability in symbolic systems, where probabilistic interactions, regeneration dynamics, and feedback loops drive the evolution of patterns. Using simulations inspired by prebiotic evolution and resource-constrained environments, we demonstrate the principles of feedback-driven stability and discuss implications for Bayesian reasoning, top-down causality, and the evolution of information.

\section{Theoretical Framework}

\subsection{Feedback Dynamics in Symbolic Systems}

Let a system consist of a population of elements \( E = \{A, B, C, \ldots\} \), where each element \( X \in E \) has a frequency \( p_X \) and a regeneration rate \( r_X \). Pairs (or higher-order combinations) of elements form compounds \( C = AB, ABC, \ldots \), characterized by their stability \( S(C) \).

\subsubsection{Regeneration Feedback Loop}

The regeneration rate \( r_X \) evolves based on the frequency of compounds containing \( X \):
\begin{equation}
r_X(t+1) = r_X(t) + \alpha \sum_{C \ni X} p(C),
\end{equation}
where \( \alpha \) is the feedback strength and \( p(C) \) is the frequency of compound \( C \). This feedback loop creates a self-reinforcing mechanism where elements contributing to stable patterns are regenerated more frequently.

\subsubsection{Interaction Probabilities}

Interactions between elements and compounds are governed by their regeneration rates and stability:
\begin{equation}
P(C = AB) \propto r_A \cdot r_B \cdot (1 + \beta \cdot S(C)),
\end{equation}
where \( \beta \) biases interactions toward stable compounds.

\subsubsection{Resource Allocation and Persistence}

Resources are allocated preferentially to stable compounds:
\begin{equation}
R(C) = \frac{S(C)}{\sum_{C'} S(C')},
\end{equation}
ensuring that persistent compounds dominate over time.

\section{Simulation Results}

We implemented the framework in a Python simulation, using symbolic populations with base elements \( A, B, C, X, Y, Z \). Key parameters include regeneration rates, interaction probabilities, and resource constraints. 

\subsection{Frequency Dynamics and Stability Emergence}

Simulations showed that stable patterns emerged as a result of feedback loops. Dominant compounds persisted across generations, and their constituent elements were regenerated at higher rates. Adjustments, such as increasing the interaction bias toward stable compounds or reducing the regeneration rate of less frequent elements, amplified these effects.

\subsection{Bayesian Reasoning in Population Dynamics}

The frequency dynamics between generations can be interpreted through Bayesian reasoning:
\begin{equation}
p(C|t+1) \propto p(C|t) \cdot \text{Fitness}(C),
\end{equation}
where fitness is proportional to stability \( S(C) \). This reflects a process where each generation updates its distribution of patterns based on environmental feedback, analogous to Bayesian inference.

\subsection{Top-Down and Bottom-Up Causality}

The framework illustrates how top-down causality arises in emergent systems. Stable patterns influence the regeneration rates of their constituent elements, thereby shaping the local interaction rules. This feedback loop creates a co-dependence between top-down (pattern-level) and bottom-up (element-level) dynamics, challenging reductionist views that prioritize microstates over macrostates~\cite{laughlin2000different}.

\section{Discussion}

\subsection{Implications for Biological and Chemical Systems}

The principles of feedback-driven stability have parallels in biological evolution and prebiotic chemistry. For example, autocatalytic cycles in chemical networks exhibit similar dynamics, where stability emerges through self-reinforcing interactions~\cite{kauffman1993origins}.

\subsection{Universal Principles of Emergence}

The framework suggests that emergent stability is a universal property of systems with probabilistic interactions, resource constraints, and feedback loops. This aligns with theories of self-organized criticality, where local rules generate global order~\cite{bak1996how}.

\subsection{Bayesian and Mystical Interpretations}

The Bayesian perspective on frequency dynamics connects mathematical reasoning to the evolution of information. Interestingly, mystical traditions such as Kabbalah and certain Indian philosophies resonate with the notion of transcendental principles guiding physical systems. However, our findings suggest that these principles are co-dependent on the dynamics of the systems they govern.

\section{Conclusion}

Feedback-driven stability offers a unifying framework for understanding pattern formation in emergent systems. By combining Bayesian reasoning with insights into top-down and bottom-up causality, this approach provides a new lens to study the evolution of information across diverse domains.

\section{Predator-Prey Dynamics in Symbolic Systems}

\subsection{Mapping Symbolic Interactions to Predator-Prey Models}

The dynamics of the symbolic system can be conceptualized using predator-prey models, where the base elements act as "prey," and the compounds they form act as "predators." The interplay between regeneration rates of base symbols and the stability of compounds creates a feedback loop analogous to classical predator-prey relationships.

In this analogy:
\begin{itemize}
    \item \textbf{Base elements} (\( A, B, C, \ldots \)) represent the prey, which are replenished through regeneration.
    \item \textbf{Compounds} (\( AB, ABC, \ldots \)) represent the predators, which consume base elements during interactions and persist based on their stability.
\end{itemize}

\subsection{Differential Equations for Predator-Prey Dynamics}

The dynamics can be expressed through a modified Lotka-Volterra system, where the populations of base elements (\( N_X \)) and compounds (\( N_C \)) evolve as:
\begin{align}
\frac{dN_X}{dt} &= r_X \cdot N_X - \sum_C P(C) \cdot N_X, \\
\frac{dN_C}{dt} &= \beta \cdot P(C) \cdot N_X - \delta \cdot N_C,
\end{align}
where:
\begin{itemize}
    \item \( r_X \): Regeneration rate of the base element \( X \) (prey).
    \item \( P(C) \): Probability of forming compound \( C \) (predator).
    \item \( \beta \): Rate of compound formation (predator growth rate).
    \item \( \delta \): Decay rate of compounds (predator mortality rate).
\end{itemize}

These equations describe a dynamic balance:
\begin{enumerate}
    \item Base elements are replenished at a rate proportional to \( r_X \), but are depleted by interactions forming compounds.
    \item Compounds grow in population as they consume base elements but decay over time due to resource competition and energy constraints.
\end{enumerate}

\subsection{Emergent Stability from Predator-Prey Feedback}

The predator-prey model inherently incorporates feedback:
\begin{itemize}
    \item \textbf{Prey Regulation}: The regeneration rates \( r_X \) dynamically adjust based on the presence of compounds, reflecting a feedback loop that maintains the population of base elements.
    \item \textbf{Predator Reinforcement}: Persistent compounds act as "successful predators," increasing the regeneration of their constituent base elements and stabilizing their own presence.
\end{itemize}

This dynamic leads to emergent stability, where certain compounds dominate due to their ability to balance resource consumption (prey depletion) and persistence (predator survival).

\subsection{Comparison to Classical Predator-Prey Systems}

Unlike traditional predator-prey models, this system includes:
\begin{itemize}
    \item \textbf{Feedback in Regeneration}: The regeneration rates \( r_X \) are influenced by the compounds they help form, creating a self-reinforcing loop absent in classical models.
    \item \textbf{Stability Bias}: Compounds with higher stability (\( S(C) \)) disproportionately affect the regeneration and interaction dynamics, skewing the system toward emergent patterns.
    \item \textbf{Multi-Prey and Multi-Predator Interactions}: Compounds can act as predators that consume multiple base elements (prey), while also being prey for higher-order interactions.
\end{itemize}

\subsection{Implications for Emergent Patterns}

The predator-prey analogy highlights:
\begin{enumerate}
    \item \textbf{Resource Constraints}: Persistent compounds stabilize by balancing consumption and regeneration, ensuring long-term survival in a resource-limited environment.
    \item \textbf{Hierarchical Emergence}: The interplay between base elements and compounds creates a hierarchical system, with dominant compounds shaping the population dynamics of their constituent elements.
    \item \textbf{Universal Applicability}: These principles extend beyond symbolic systems, providing insights into ecological networks, chemical reaction systems, and evolutionary dynamics.
\end{enumerate}

The predator-prey perspective complements the Bayesian and top-down causality interpretations, offering an alternative lens to understand the co-dependence of local interactions and global stability.

\section{Incorporating Resources and Energy into Symbolic Systems}

\subsection{Energy as a Limiting Factor in Interactions}

In real-world systems, interactions between elements and the persistence of patterns are often constrained by the availability of energy. To incorporate this into the symbolic framework, we introduce an explicit energy model, where:
\begin{itemize}
    \item \textbf{Total Energy}: The system starts with an initial energy pool \( E(t) \), which is replenished at each generation by an external energy source.
    \item \textbf{Energy for Interactions}: Each interaction consumes a fixed amount of energy proportional to the size of the compound formed.
    \item \textbf{Energy Recycling}: Decaying compounds release a fraction of their energy back into the system, reflecting energy recycling dynamics.
\end{itemize}

\subsection{Mathematical Model for Energy Dynamics}

The energy dynamics are modeled as follows:
\begin{align}
E(t+1) &= E(t) + \epsilon_{\text{replenish}} - \epsilon_{\text{interactions}} + \epsilon_{\text{recycled}}, \\
\epsilon_{\text{interactions}} &= \sum_{i=1}^{N} \epsilon_{\text{cost}}(C_i), \\
\epsilon_{\text{recycled}} &= \sum_{j=1}^{M} \gamma \cdot S(C_j),
\end{align}
where:
\begin{itemize}
    \item \( E(t) \): Total energy at generation \( t \).
    \item \( \epsilon_{\text{replenish}} \): Energy added externally to the system at each generation.
    \item \( \epsilon_{\text{interactions}} \): Energy consumed during interactions.
    \item \( \epsilon_{\text{cost}}(C_i) \): Energy cost of forming compound \( C_i \), proportional to its size or complexity.
    \item \( \epsilon_{\text{recycled}} \): Energy recycled from decaying compounds.
    \item \( \gamma \): Fraction of stability \( S(C_j) \) recycled into usable energy.
\end{itemize}

\subsection{Resource Constraints on Regeneration}

In addition to energy, resources are treated as a separate pool \( R(t) \), representing the raw materials needed to regenerate base elements and form compounds:
\begin{align}
R(t+1) &= R(t) + \rho_{\text{replenish}} - \rho_{\text{consumed}}, \\
\rho_{\text{consumed}} &= \sum_{C} \lambda \cdot \text{size}(C),
\end{align}
where:
\begin{itemize}
    \item \( R(t) \): Total resources at generation \( t \).
    \item \( \rho_{\text{replenish}} \): Resources replenished at each generation.
    \item \( \rho_{\text{consumed}} \): Resources consumed during interactions, proportional to compound size.
    \item \( \lambda \): Resource cost per unit of compound size.
\end{itemize}

\subsection{Energy and Resource Allocation for Patterns}

Energy and resources are allocated based on compound stability \( S(C) \):
\begin{align}
\text{Energy Share } \epsilon(C) &= \frac{S(C)}{\sum_{C'} S(C')}, \\
\text{Resource Share } \rho(C) &= \frac{S(C)}{\sum_{C'} S(C')}.
\end{align}
Compounds with higher stability are allocated a larger share, promoting their persistence and dominance in the population.

\subsection{Emergent Dynamics with Energy and Resources}

Adding energy and resources introduces new constraints and opportunities:
\begin{enumerate}
    \item \textbf{Resource Competition}: Compounds compete for limited resources and energy, favoring the most stable patterns.
    \item \textbf{Energy Recycling}: The recycling of energy from decaying compounds creates a feedback loop, ensuring long-term sustainability of the system.
    \item \textbf{Persistence of Stable Patterns}: Stability directly influences energy and resource allocation, reinforcing the dominance of persistent patterns.
\end{enumerate}

\subsection{Comparison to Real-World Systems}

This model mirrors real-world phenomena:
\begin{itemize}
    \item \textbf{Biological Systems}: Energy and nutrient cycles in ecosystems are analogous to energy recycling and resource replenishment in the model.
    \item \textbf{Chemical Systems}: Reaction networks depend on the availability of energy to form and break bonds, analogous to the energy costs of forming compounds.
    \item \textbf{Information Systems}: Computational processes constrained by energy and resource availability exhibit similar dynamics, with stable patterns corresponding to persistent solutions.
\end{itemize}

\subsection{Implications for Stability and Complexity}

The incorporation of energy and resources enhances the framework by:
\begin{enumerate}
    \item Providing a mechanism to regulate the balance between base elements and compounds.
    \item Highlighting the role of external inputs in driving complexity and stability.
    \item Offering insights into how energy constraints shape emergent behavior in both physical and abstract systems.
\end{enumerate}

\section{From Pattern Systems to Cellular Automata: Evolution Driven by Stability Imbalances}

\subsection{Simple Pattern Systems and Stability-Driven Selection}

The foundation of this work begins with a simple symbolic system where basic elements \( E = \{A, B, C, \ldots\} \) interact to form compounds. These compounds are characterized by stability values \( S(C) \), which determine their persistence over multiple generations. The system evolves through:
\begin{enumerate}
    \item \textbf{Regeneration}: Base elements are replenished in each generation to ensure a continuous pool for interactions.
    \item \textbf{Interactions}: Elements combine probabilistically, with stronger attractive forces forming more stable compounds.
    \item \textbf{Selection Pressure}: More stable compounds persist and contribute disproportionately to the population, creating a feedback loop that reinforces stability.
\end{enumerate}

This process creates an emergent hierarchy of patterns, where stable configurations dominate over time. Despite the absence of explicit replication or inheritance, the system exhibits fitness-proportionate selection, where stability biases the frequency of patterns in subsequent generations.

\subsection{Transition to Cellular Automata}

Building on the dynamics of symbolic pattern systems, we extend the framework to evolve rules for cellular automata (CA). The core idea is that:
\begin{enumerate}
    \item Symbols \( E = \{A, B, C, \ldots\} \) serve as the building blocks for local state transitions in a CA.
    \item Interactions between symbols form transition rules \( (s_{i-1}, s_i, s_{i+1}) \rightarrow s_i' \), analogous to the formation of compounds in the pattern system.
    \item Stability imbalances determine the likelihood of forming certain rules, with more stable rules persisting over multiple generations.
\end{enumerate}

\subsection{Mechanism of CA Rule Evolution}

In the CA framework, each generation involves:
\begin{enumerate}
    \item \textbf{Rule Formation}: Symbols interact probabilistically to create transition rules. The stability of a rule is analogous to the stability of compounds in the pattern system.
    \item \textbf{Rule Evaluation}: The generated rules are applied to a 1D CA, evolving its state over several generations.
    \item \textbf{Feedback and Reinforcement}: Rules that produce stable or interesting CA dynamics (e.g., persistent patterns or oscillations) are reinforced, increasing the frequency of their constituent symbols.
    \item \textbf{Regeneration}: Base symbols are replenished to prevent convergence and maintain diversity, enabling new rule formation.
\end{enumerate}

\subsection{Stability and Selection Pressure in CA Evolution}

As in the pattern system, stability drives selection pressure in the CA. Rules that lead to persistent or complex patterns are favored, while trivial or unstable rules dissipate. Over time, the system self-organizes to favor:
\begin{itemize}
    \item \textbf{Persistent Rules}: Rules that maintain stable structures (e.g., gliders or oscillators).
    \item \textbf{Hierarchical Complexity}: Emergent behavior arises from the interplay of rules, with stable rules dominating and influencing the evolution of less stable ones.
\end{itemize}

\subsection{Comparison of Pattern Systems and CA Evolution}

The transition from simple pattern systems to cellular automata evolution highlights a shared foundation:
\begin{enumerate}
    \item \textbf{Local Interactions}: Both systems rely on probabilistic local interactions driven by stability.
    \item \textbf{Selection Pressure}: Stability biases the persistence of patterns or rules, mimicking fitness-driven selection.
    \item \textbf{Emergent Behavior}: Both systems exhibit emergent order over generations, despite starting from random initial conditions.
\end{enumerate}

However, CA evolution introduces an additional layer of complexity:
\begin{itemize}
    \item Rules, rather than patterns, act as the units of persistence, creating a meta-level of evolution.
    \item The feedback loop involves not just the frequency of symbols but also the evaluation of their derived rules in the context of CA dynamics.
\end{itemize}

\subsection{Implications and Applications}

This approach demonstrates how local stability imbalances can drive the evolution of systems from simple patterns to complex dynamical behaviors. The principles outlined here have potential applications in:
\begin{enumerate}
    \item \textbf{Prebiotic Chemistry}: Understanding how simple molecules evolved into complex reaction networks.
    \item \textbf{Computational Systems}: Designing self-evolving algorithms for cellular automata or other computational models.
    \item \textbf{Complex Systems Theory}: Exploring how emergent order arises from local interactions and selection pressures.
\end{enumerate}

By linking the evolution of symbolic patterns to cellular automata, this work bridges the gap between abstract models of stability and real-world systems characterized by dynamic interactions and emergent complexity.

\section{Evolution of Rules through Stability Imbalances}

The formation of rules, such as the example generated for Wolfram’s cellular automaton, arises from the interplay of stability imbalances, regeneration, and selection dynamics across multiple generations. This section explains how such rules evolve, emphasizing the emergent nature of their formation.

\subsection{Initial Conditions and Stability Imbalances}

The system begins with:
\begin{itemize}
    \item \textbf{Symbols}: The initial state is a random sequence composed of symbols \( \{0, 1\} \), representing the possible states of the system.
    \item \textbf{Stability Values}: Each neighborhood (e.g., \((0, 0, 1)\), \((1, 1, 0)\)) has an associated stability value, defined in a stability dictionary. These values determine the likelihood that a given neighborhood will persist or generate a specific output (\(0\) or \(1\)).
\end{itemize}

Neighborhoods with higher stability values are more likely to influence the system's evolution, while neighborhoods with lower stability values are less likely to contribute to rule formation.

\subsection{Dynamics Across Generations}

The system evolves dynamically through the following steps:
\begin{enumerate}
    \item \textbf{Neighborhood Interactions}: At each generation, the current state defines neighborhoods for every cell. The output of each neighborhood is determined probabilistically, with higher stability values biasing the result toward \(1\).
    \item \textbf{Regeneration and Feedback}: Over successive generations:
    \begin{itemize}
        \item High-stability neighborhoods (e.g., \((0, 0, 1)\), \((1, 0, 1)\)) are more likely to generate \(1\), reinforcing their persistence in subsequent states.
        \item Low-stability neighborhoods (e.g., \((0, 0, 0)\)) default to \(0\), gradually reducing their impact on the system.
    \end{itemize}
    \item \textbf{Emergent Rule Formation}: After multiple generations, the system exhibits consistent behavior, effectively encoding a rule that favors high-stability neighborhoods while suppressing low-stability ones.
\end{enumerate}

\subsection{Example Rule Formation}

Consider the following generated rule:
\[
\text{Generated Rule: } \{(0, 0, 1): 1, (0, 1, 1): 1, (1, 0, 0): 1, (1, 0, 1): 1, (1, 1, 0): 1, (1, 1, 1): 1\}
\]

\paragraph{Key Observations:}
\begin{itemize}
    \item \textbf{High Stability for Black-Dominant Neighborhoods}: Neighborhoods with at least one \(1\) (e.g., \((0, 0, 1)\), \((1, 1, 0)\)) are more likely to produce \(1\), reflecting the higher stability values assigned to these configurations.
    \item \textbf{Low Stability for White-Only Neighborhoods}: Neighborhoods like \((0, 0, 0)\) have lower stability values, resulting in default outputs of \(0\). Over generations, their impact diminishes.
    \item \textbf{Rule Emergence}: The generated rule aligns with Wolfram’s behavior, where a cell becomes black (\(1\)) if either of its neighbors is black in the previous generation.
\end{itemize}

\subsection{Role of Stability Imbalances}

Stability imbalances act as a form of selection pressure, driving the evolution of rules through:
\begin{itemize}
    \item \textbf{Bias Toward High-Stability Patterns}: Neighborhoods with higher stability values are reinforced across generations, creating feedback loops that amplify their presence in the population.
    \item \textbf{Suppression of Low-Stability Patterns}: Low-stability neighborhoods contribute less to the population over time, enabling the dominance of high-stability configurations.
\end{itemize}

\subsection{Comparison to Evolutionary Principles}

The process mirrors evolutionary dynamics:
\begin{enumerate}
    \item \textbf{Variation}: Random initial states introduce diverse neighborhoods, providing a "pool" of potential traits.
    \item \textbf{Selection}: Stability values bias the persistence of certain neighborhoods, akin to natural selection.
    \item \textbf{Inheritance}: High-stability neighborhoods persist across generations, reinforcing their influence.
    \item \textbf{Emergence}: Over time, consistent rules emerge, not because they are predefined but because they align with the system's inherent stability dynamics.
\end{enumerate}

\subsection{Conclusion}

Through stability imbalances, regeneration, and feedback loops, the system evolves from a random initial state to a structured rule set that encodes consistent behavior. The resulting rule, shaped by probabilistic interactions and selection dynamics, exemplifies the emergence of complexity from simple underlying principles.



%%%%%%%%%%%%%%%%%%%%%%%%%%%%%%%%%%%%%%%%%%
\section{Materials and Methods}

Materials and Methods should be described with sufficient details to allow others to replicate and build on published results. Please note that publication of your manuscript implicates that you must make all materials, data, computer code, and protocols associated with the publication available to readers. Please disclose at the submission stage any restrictions on the availability of materials or information. New methods and protocols should be described in detail while well-established methods can be briefly described and appropriately cited.

Research manuscripts reporting large datasets that are deposited in a publicly avail-able database should specify where the data have been deposited and provide the relevant accession numbers. If the accession numbers have not yet been obtained at the time of submission, please state that they will be provided during review. They must be provided prior to publication.

Interventionary studies involving animals or humans, and other studies require ethical approval must list the authority that provided approval and the corresponding ethical approval code.
\begin{quote}
This is an example of a quote.
\end{quote}

%%%%%%%%%%%%%%%%%%%%%%%%%%%%%%%%%%%%%%%%%%
\section{Results}

This section may be divided by subheadings. It should provide a concise and precise description of the experimental results, their interpretation as well as the experimental conclusions that can be drawn.
\subsection{Subsection}
\subsubsection{Subsubsection}

Bulleted lists look like this:
\begin{itemize}
\item	First bullet;
\item	Second bullet;
\item	Third bullet.
\end{itemize}

Numbered lists can be added as follows:
\begin{enumerate}
\item	First item; 
\item	Second item;
\item	Third item.
\end{enumerate}

The text continues here. 

\subsection{Figures, Tables and Schemes}

All figures and tables should be cited in the main text as Figure~\ref{fig1}, Table~\ref{tab1}, etc.

\begin{figure}[H]
\includegraphics[width=10.5 cm]{Definitions/logo-mdpi}
\caption{This is a figure. Schemes follow the same formatting. If there are multiple panels, they should be listed as: (\textbf{a}) Description of what is contained in the first panel. (\textbf{b}) Description of what is contained in the second panel. Figures should be placed in the main text near to the first time they are cited. A caption on a single line should be centered.\label{fig1}}
\end{figure}   
\unskip

\begin{table}[H] 
\caption{This is a table caption. Tables should be placed in the main text near to the first time they are~cited.\label{tab1}}
%\newcolumntype{C}{>{\centering\arraybackslash}X}
\begin{tabularx}{\textwidth}{CCC}
\toprule
\textbf{Title 1}	& \textbf{Title 2}	& \textbf{Title 3}\\
\midrule
Entry 1		& Data			& Data\\
Entry 2		& Data			& Data \textsuperscript{1}\\
\bottomrule
\end{tabularx}
\noindent{\footnotesize{\textsuperscript{1} Tables may have a footer.}}
\end{table}

The text continues here (Figure~\ref{fig2} and Table~\ref{tab2}).

% Example of a figure that spans the whole page width. The same concept works for tables, too.
\begin{figure}[H]
\begin{adjustwidth}{-\extralength}{0cm}
\centering
\includegraphics[width=15.5cm]{Definitions/logo-mdpi}
\end{adjustwidth}
\caption{This is a wide figure.\label{fig2}}
\end{figure}  

\begin{table}[H]
\caption{This is a wide table.\label{tab2}}
	\begin{adjustwidth}{-\extralength}{0cm}
%		\newcolumntype{C}{>{\centering\arraybackslash}X}
		\begin{tabularx}{\fulllength}{CCCC}
			\toprule
			\textbf{Title 1}	& \textbf{Title 2}	& \textbf{Title 3}     & \textbf{Title 4}\\
			\midrule
\multirow[m]{3}{*}{Entry 1 *}	& Data			& Data			& Data\\
			  	                   & Data			& Data			& Data\\
			             	      & Data			& Data			& Data\\
                   \midrule
\multirow[m]{3}{*}{Entry 2}    & Data			& Data			& Data\\
			  	                  & Data			& Data			& Data\\
			             	     & Data			& Data			& Data\\
                   \midrule
\multirow[m]{3}{*}{Entry 3}    & Data			& Data			& Data\\
			  	                 & Data			& Data			& Data\\
			             	    & Data			& Data			& Data\\
                  \midrule
\multirow[m]{3}{*}{Entry 4}   & Data			& Data			& Data\\
			  	                 & Data			& Data			& Data\\
			             	    & Data			& Data			& Data\\
			\bottomrule
		\end{tabularx}
	\end{adjustwidth}
	\noindent{\footnotesize{* Tables may have a footer.}}
\end{table}

%\begin{listing}[H]
%\caption{Title of the listing}
%\rule{\columnwidth}{1pt}
%\raggedright Text of the listing. In font size footnotesize, small, or normalsize. Preferred format: left aligned and single spaced. Preferred border format: top border line and bottom border line.
%\rule{\columnwidth}{1pt}
%\end{listing}

Text.

Text.

\subsection{Formatting of Mathematical Components}

This is the example 1 of equation:
\begin{linenomath}
\begin{equation}
a = 1,
\end{equation}
\end{linenomath}
the text following an equation need not be a new paragraph. Please punctuate equations as regular text.
%% If the documentclass option "submit" is chosen, please insert a blank line before and after any math environment (equation and eqnarray environments). This ensures correct linenumbering. The blank line should be removed when the documentclass option is changed to "accept" because the text following an equation should not be a new paragraph.

This is the example 2 of equation:
\begin{adjustwidth}{-\extralength}{0cm}
\begin{equation}
a = b + c + d + e + f + g + h + i + j + k + l + m + n + o + p + q + r + s + t + u + v + w + x + y + z
\end{equation}
\end{adjustwidth}

% Example of a page in landscape format (with table and table footnote).
%\startlandscape
%\begin{table}[H] %% Table in wide page
%\caption{This is a very wide table.\label{tab3}}
%	\begin{tabularx}{\textwidth}{CCCC}
%		\toprule
%		\textbf{Title 1}	& \textbf{Title 2}	& \textbf{Title 3}	& \textbf{Title 4}\\
%		\midrule
%		Entry 1		& Data			& Data			& This cell has some longer content that runs over two lines.\\
%		Entry 2		& Data			& Data			& Data\textsuperscript{1}\\
%		\bottomrule
%	\end{tabularx}
%	\begin{adjustwidth}{+\extralength}{0cm}
%		\noindent\footnotesize{\textsuperscript{1} This is a table footnote.}
%	\end{adjustwidth}
%\end{table}
%\finishlandscape


Please punctuate equations as regular text. Theorem-type environments (including propositions, lemmas, corollaries etc.) can be formatted as follows:
%% Example of a theorem:
\begin{Theorem}
Example text of a theorem.
\end{Theorem}

The text continues here. Proofs must be formatted as follows:

%% Example of a proof:
\begin{proof}[Proof of Theorem 1]
Text of the proof. Note that the phrase ``of Theorem 1'' is optional if it is clear which theorem is being referred to.
\end{proof}
The text continues here.

%%%%%%%%%%%%%%%%%%%%%%%%%%%%%%%%%%%%%%%%%%
\section{Discussion}

Authors should discuss the results and how they can be interpreted from the perspective of previous studies and of the working hypotheses. The findings and their implications should be discussed in the broadest context possible. Future research directions may also be highlighted.

%%%%%%%%%%%%%%%%%%%%%%%%%%%%%%%%%%%%%%%%%%
\section{Conclusions}

This section is not mandatory, but can be added to the manuscript if the discussion is unusually long or complex.

%%%%%%%%%%%%%%%%%%%%%%%%%%%%%%%%%%%%%%%%%%
\section{Patents}

This section is not mandatory, but may be added if there are patents resulting from the work reported in this manuscript.

%%%%%%%%%%%%%%%%%%%%%%%%%%%%%%%%%%%%%%%%%%
\vspace{6pt} 

%%%%%%%%%%%%%%%%%%%%%%%%%%%%%%%%%%%%%%%%%%
%% optional
%\supplementary{The following supporting information can be downloaded at:  \linksupplementary{s1}, Figure S1: title; Table S1: title; Video S1: title.}

% Only for journal Methods and Protocols:
% If you wish to submit a video article, please do so with any other supplementary material.
% \supplementary{The following supporting information can be downloaded at: \linksupplementary{s1}, Figure S1: title; Table S1: title; Video S1: title. A supporting video article is available at doi: link.}

% Only for journal Hardware:
% If you wish to submit a video article, please do so with any other supplementary material.
% \supplementary{The following supporting information can be downloaded at: \linksupplementary{s1}, Figure S1: title; Table S1: title; Video S1: title.\vspace{6pt}\\
%\begin{tabularx}{\textwidth}{lll}
%\toprule
%\textbf{Name} & \textbf{Type} & \textbf{Description} \\
%\midrule
%S1 & Python script (.py) & Script of python source code used in XX \\
%S2 & Text (.txt) & Script of modelling code used to make Figure X \\
%S3 & Text (.txt) & Raw data from experiment X \\
%S4 & Video (.mp4) & Video demonstrating the hardware in use \\
%... & ... & ... \\
%\bottomrule
%\end{tabularx}
%}

%%%%%%%%%%%%%%%%%%%%%%%%%%%%%%%%%%%%%%%%%%
\authorcontributions{For research articles with several authors, a short paragraph specifying their individual contributions must be provided. The following statements should be used ``Conceptualization, X.X. and Y.Y.; methodology, X.X.; software, X.X.; validation, X.X., Y.Y. and Z.Z.; formal analysis, X.X.; investigation, X.X.; resources, X.X.; data curation, X.X.; writing---original draft preparation, X.X.; writing---review and editing, X.X.; visualization, X.X.; supervision, X.X.; project administration, X.X.; funding acquisition, Y.Y. All authors have read and agreed to the published version of the manuscript.'', please turn to the  \href{http://img.mdpi.org/data/contributor-role-instruction.pdf}{CRediT taxonomy} for the term explanation. Authorship must be limited to those who have contributed substantially to the work~reported.}

\funding{Please add: ``This research received no external funding'' or ``This research was funded by NAME OF FUNDER grant number XXX.'' and  and ``The APC was funded by XXX''. Check carefully that the details given are accurate and use the standard spelling of funding agency names at \url{https://search.crossref.org/funding}, any errors may affect your future funding.}

\institutionalreview{In this section, you should add the Institutional Review Board Statement and approval number, if relevant to your study. You might choose to exclude this statement if the study did not require ethical approval. Please note that the Editorial Office might ask you for further information. Please add “The study was conducted in accordance with the Declaration of Helsinki, and approved by the Institutional Review Board (or Ethics Committee) of NAME OF INSTITUTE (protocol code XXX and date of approval).” for studies involving humans. OR “The animal study protocol was approved by the Institutional Review Board (or Ethics Committee) of NAME OF INSTITUTE (protocol code XXX and date of approval).” for studies involving animals. OR “Ethical review and approval were waived for this study due to REASON (please provide a detailed justification).” OR “Not applicable” for studies not involving humans or animals.}

\informedconsent{Any research article describing a study involving humans should contain this statement. Please add ``Informed consent was obtained from all subjects involved in the study.'' OR ``Patient consent was waived due to REASON (please provide a detailed justification).'' OR ``Not applicable'' for studies not involving humans. You might also choose to exclude this statement if the study did not involve humans.

Written informed consent for publication must be obtained from participating patients who can be identified (including by the patients themselves). Please state ``Written informed consent has been obtained from the patient(s) to publish this paper'' if applicable.}

\dataavailability{We encourage all authors of articles published in MDPI journals to share their research data. In this section, please provide details regarding where data supporting reported results can be found, including links to publicly archived datasets analyzed or generated during the study. Where no new data were created, or where data is unavailable due to privacy or ethical restrictions, a statement is still required. Suggested Data Availability Statements are available in section ``MDPI Research Data Policies'' at \url{https://www.mdpi.com/ethics}.} 

% Only for journal Nursing Reports
%\publicinvolvement{Please describe how the public (patients, consumers, carers) were involved in the research. Consider reporting against the GRIPP2 (Guidance for Reporting Involvement of Patients and the Public) checklist. If the public were not involved in any aspect of the research add: ``No public involvement in any aspect of this research''.}

% Only for journal Nursing Reports
%\guidelinesstandards{Please add a statement indicating which reporting guideline was used when drafting the report. For example, ``This manuscript was drafted against the XXX (the full name of reporting guidelines and citation) for XXX (type of research) research''. A complete list of reporting guidelines can be accessed via the equator network: \url{https://www.equator-network.org/}.}

% Only for journal Nursing Reports
%\useofartificialintelligence{Please describe in detail any and all uses of artificial intelligence (AI) or AI-assisted tools used in the preparation of the manuscript. This may include, but is not limited to, language translation, language editing and grammar, or generating text. Alternatively, please state that “AI or AI-assisted tools were not used in drafting any aspect of this manuscript”.}

\acknowledgments{In this section you can acknowledge any support given which is not covered by the author contribution or funding sections. This may include administrative and technical support, or donations in kind (e.g., materials used for experiments).}

\conflictsofinterest{Declare conflicts of interest or state ``The authors declare no conflicts of interest.'' Authors must identify and declare any personal circumstances or interest that may be perceived as inappropriately influencing the representation or interpretation of reported research results. Any role of the funders in the design of the study; in the collection, analyses or interpretation of data; in the writing of the manuscript; or in the decision to publish the results must be declared in this section. If there is no role, please state ``The funders had no role in the design of the study; in the collection, analyses, or interpretation of data; in the writing of the manuscript; or in the decision to publish the results''.} 

%%%%%%%%%%%%%%%%%%%%%%%%%%%%%%%%%%%%%%%%%%
%% Optional

%% Only for journal Encyclopedia
%\entrylink{The Link to this entry published on the encyclopedia platform.}

\abbreviations{Abbreviations}{
The following abbreviations are used in this manuscript:\\

\noindent 
\begin{tabular}{@{}ll}
MDPI & Multidisciplinary Digital Publishing Institute\\
DOAJ & Directory of open access journals\\
TLA & Three letter acronym\\
LD & Linear dichroism
\end{tabular}
}

%%%%%%%%%%%%%%%%%%%%%%%%%%%%%%%%%%%%%%%%%%
%% Optional
\appendixtitles{no} % Leave argument "no" if all appendix headings stay EMPTY (then no dot is printed after "Appendix A"). If the appendix sections contain a heading then change the argument to "yes".
\appendixstart
\appendix
\section[\appendixname~\thesection]{}
\subsection[\appendixname~\thesubsection]{}
The appendix is an optional section that can contain details and data supplemental to the main text---for example, explanations of experimental details that would disrupt the flow of the main text but nonetheless remain crucial to understanding and reproducing the research shown; figures of replicates for experiments of which representative data are shown in the main text can be added here if brief, or as Supplementary Data. Mathematical proofs of results not central to the paper can be added as an appendix.

\begin{table}[H] 
\caption{This is a table caption.\label{tab5}}
\newcolumntype{C}{>{\centering\arraybackslash}X}
\begin{tabularx}{\textwidth}{CCC}
\toprule
\textbf{Title 1}	& \textbf{Title 2}	& \textbf{Title 3}\\
\midrule
Entry 1		& Data			& Data\\
Entry 2		& Data			& Data\\
\bottomrule
\end{tabularx}
\end{table}

\section[\appendixname~\thesection]{}
All appendix sections must be cited in the main text. In the appendices, Figures, Tables, etc. should be labeled, starting with ``A''---e.g., Figure A1, Figure A2, etc.

%%%%%%%%%%%%%%%%%%%%%%%%%%%%%%%%%%%%%%%%%%
\begin{adjustwidth}{-\extralength}{0cm}
%\printendnotes[custom] % Un-comment to print a list of endnotes

\reftitle{References}

% Please provide either the correct journal abbreviation (e.g. according to the “List of Title Word Abbreviations” http://www.issn.org/services/online-services/access-to-the-ltwa/) or the full name of the journal.
% Citations and References in Supplementary files are permitted provided that they also appear in the reference list here. 

%=====================================
% References, variant A: external bibliography
%=====================================
%\bibliography{your_external_BibTeX_file}

%=====================================
% References, variant B: internal bibliography
%=====================================
\begin{thebibliography}{999}
% Reference 1
\bibitem[Author1(year)]{ref-journal}
Author~1, T. The title of the cited article. {\em Journal Abbreviation} {\bf 2008}, {\em 10}, 142--149.
% Reference 2
\bibitem[Author2(year)]{ref-book1}
Author~2, L. The title of the cited contribution. In {\em The Book Title}; Editor 1, F., Editor 2, A., Eds.; Publishing House: City, Country, 2007; pp. 32--58.
% Reference 3
\bibitem[Author3(year)]{ref-book2}
Author 1, A.; Author 2, B. \textit{Book Title}, 3rd ed.; Publisher: Publisher Location, Country, 2008; pp. 154--196.
% Reference 4
\bibitem[Author4(year)]{ref-unpublish}
Author 1, A.B.; Author 2, C. Title of Unpublished Work. \textit{Abbreviated Journal Name} year, \textit{phrase indicating stage of publication (submitted; accepted; in press)}.
% Reference 5
\bibitem[Author5(year)]{ref-communication}
Author 1, A.B. (University, City, State, Country); Author 2, C. (Institute, City, State, Country). Personal communication, 2012.
% Reference 6
\bibitem[Author6(year)]{ref-proceeding}
Author 1, A.B.; Author 2, C.D.; Author 3, E.F. Title of presentation. In Proceedings of the Name of the Conference, Location of Conference, Country, Date of Conference (Day Month Year); Abstract Number (optional), Pagination (optional).
% Reference 7
\bibitem[Author7(year)]{ref-thesis}
Author 1, A.B. Title of Thesis. Level of Thesis, Degree-Granting University, Location of University, Date of Completion.
% Reference 8
\bibitem[Author8(year)]{ref-url}
Title of Site. Available online: URL (accessed on Day Month Year).
\end{thebibliography}

% If authors have biography, please use the format below
%\section*{Short Biography of Authors}
%\bio
%{\raisebox{-0.35cm}{\includegraphics[width=3.5cm,height=5.3cm,clip,keepaspectratio]{Definitions/author1.pdf}}}
%{\textbf{Firstname Lastname} Biography of first author}
%
%\bio
%{\raisebox{-0.35cm}{\includegraphics[width=3.5cm,height=5.3cm,clip,keepaspectratio]{Definitions/author2.jpg}}}
%{\textbf{Firstname Lastname} Biography of second author}

% For the MDPI journals use author-date citation, please follow the formatting guidelines on http://www.mdpi.com/authors/references
% To cite two works by the same author: \citeauthor{ref-journal-1a} (\citeyear{ref-journal-1a}, \citeyear{ref-journal-1b}). This produces: Whittaker (1967, 1975)
% To cite two works by the same author with specific pages: \citeauthor{ref-journal-3a} (\citeyear{ref-journal-3a}, p. 328; \citeyear{ref-journal-3b}, p.475). This produces: Wong (1999, p. 328; 2000, p. 475)

%%%%%%%%%%%%%%%%%%%%%%%%%%%%%%%%%%%%%%%%%%
%% for journal Sci
%\reviewreports{\\
%Reviewer 1 comments and authors’ response\\
%Reviewer 2 comments and authors’ response\\
%Reviewer 3 comments and authors’ response
%}
%%%%%%%%%%%%%%%%%%%%%%%%%%%%%%%%%%%%%%%%%%
\PublishersNote{}
\end{adjustwidth}
\end{document}

